\documentclass[11pt]{article}

\usepackage{parskip}
\usepackage{fullpage}
\usepackage{graphicx}
\usepackage{amsmath}
\usepackage{amssymb}
\usepackage{amsthm}
\usepackage{fancyvrb}
\usepackage{algorithm}
\usepackage[noend]{algpseudocode}
\theoremstyle{definition}
\newtheorem{definition}{Definition}
\theoremstyle{plain}
\newtheorem{theorem}{Theorem}
\newtheorem{lemma}{Lemma}
\newtheorem{example}{Example}[section]
\parindent0in
\pagestyle{plain}
\thispagestyle{plain}


%% UPDATE MACRO DEFINITIONS %%
\newcommand{\myname}{Scribed By: Manish(2018101073)}
\newcommand{\mynami}{Akshat Goyal(2018101075)}
\newcommand{\mynamj}{Dixit Kumar Garg(2018101077)}
\newcommand{\mynamk}{Gadela Kesav(2018101079)}
\newcommand{\assignment}{Lecture \#7}
\newcommand{\duedate}{18.02.2019}
\newtheorem*{thmtype}{Outline}
\newcommand{\statement}{
We introduce Reinforcement Learning with the Multi Arm Bandit (MAB).
}

\begin{document}

\textbf{IIIT Hyderabad}\hfill\\[0.01in]
\textbf{-}\hfill\textbf{\myname}\\[0.01in]
\textbf{-}\hfill\textbf{\mynami}\\[0.01in]
\textbf{-}\hfill\textbf{\mynamj}\\[0.01in]
\textbf{-}\hfill\textbf{\mynamk}\\[0.01in]
\textbf{Linear Algebra (MA3.101)}\hfill\textbf{\assignment}\\[0.01in]
\textbf{Instructor: Dr.\ Naresh Manwani}\hfill\textbf{\duedate}\\
\smallskip\hrule\bigskip

\begin{thmtype}
\statement
\end{thmtype}

%%% Start writing and defining sections and subsections here
\\
\section{Spanning Set}
\begin{lemma}
Let $S_1$ be a spanning set of $\mathbf{W}$ \subset $\mathbf{V}$. Suppose $S_2$ spans $S_1$ , then $S_2$ spans $\mathbf{W}$.
\end{lemma}
\begin{proof}
$$ \mathbf{W} \subset \text{span($S_1$)}$$
$$\text{$S_1$} \subset \text{span($S_2$)}$$
$$\text{span(span($S_2$))=span($S_2$) (by definition of span)}$$
$\therefore$ $S_2$ spans $\mathbf{W}$
\end{proof}

\begin{lemma}
Let S=\{$\vec{a_1},\dots,\vec{a_n}\} \text{ be a spanning set of \mathbf{W}. } Then\ S^{'}\ =\ S\backslash\{\vec{a_i}\} \cup \ \{c\vec{a_i}+\Sigma^n_{j=1 j \neq i}\vec{a_j}\},\\c \neq 0,c \in \mathbf{F}\ spans\ \mathbf{W}.$
\end{lemma}
\begin{proof}
To prove it,it's sufficient to show that $S^'$ spans S by using previous lemma.
$$\text{Clearly $\vec{a_j}$_{\ j \neq i}, \in\ $S^'$ $\subset$ span{$S^'$}}$$
$$\text{Let $\vec{a^{'}} = \{c\vec{a_i}+\Sigma^n_{j=1 j \neq i}\vec{a_j}\} \in S^{'} $}$$
$$\text{also $\vec{a_i} = c^{-1}(\vec{a^{'}}-\Sigma^n_{j=1 j \neq i}c_{j}\vec{a_{j}})\ as\ c \neq 0,c_{j} \in \mathbf{F}$}$$
$$\text{as $\vec{a_i}$ can be written as linear combination of vectors $\in$ $S^'$}$$
$$\text{$\therefore$ $\vec{a_i}$ $\in$ span($S^{'}$)(by def. of span)}$$
$$\text{$\Rightarrow$ $S^'$ spans S}$$
$\therefore$by using previous lemma
$S^'$ spans $\mathbf{W}$
\end{proof}
\end{document}