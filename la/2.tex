% \documentclass[11pt]{article}
% \usepackage{fullpage}
% \usepackage{parskip}
% \usepackage{graphicx}
% \usepackage{amsmath}
% \usepackage{comment}
% \usepackage{amssymb}
% \usepackage{amsthm}
% \usepackage{fancyvrb}
% \usepackage{algorithm}
% \usepackage[noend]{algpseudocode}
% \theoremstyle{definition}
% \newtheorem{definition}{Definition}
% \theoremstyle{plain}
% \newtheorem{theorem}{Theorem}
% \newtheorem{lemma}{Lemma}
% \newtheorem{example}{Example}[section]

% \parindent0in
% \pagestyle{plain}
% \thispagestyle{plain}
                
\documentclass[11pt]{article}

\usepackage{parskip}
\usepackage{fullpage}
\usepackage{graphicx}
\usepackage{amsmath}
\usepackage{amssymb}
\usepackage{amsthm}
\usepackage{fancyvrb}
\usepackage{algorithm}
\usepackage[noend]{algpseudocode}
\theoremstyle{definition}
\newtheorem{definition}{Definition}
\theoremstyle{plain}
\newtheorem{theorem}{Theorem}
\newtheorem{lemma}{Lemma}
\newtheorem{example}{Example}[section]
\parindent0in
\pagestyle{plain}
\thispagestyle{plain}
%% UPDATE MACRO DEFINITIONS %%
\newcommand{\myname}{Scribed By: Manish(2018101073)}
\newcommand{\mynami}{Akshat Goyal(2018101075)}
\newcommand{\mynamj}{Dixit Kumar Garg(2018101077)}
\newcommand{\mynamk}{Gadela Kesav(2018101079)}
\newcommand{\assignment}{Lecture \#7}
\newcommand{\duedate}{18.02.2019}
\newtheorem*{thmtype}{Outline}
\newcommand{\statement}{
We introduce Reinforcement Learning with the Multi Arm Bandit (MAB).
}

\begin{document}

\textbf{IIIT Hyderabad}\hfill\\[0.01in]
\textbf{-}\hfill\textbf{\myname}\\[0.01in]
\textbf{-}\hfill\textbf{\mynami}\\[0.01in]
\textbf{-}\hfill\textbf{\mynamj}\\[0.01in]
\textbf{-}\hfill\textbf{\mynamk}\\[0.01in]
\textbf{Linear Algebra (MA3.101)}\hfill\textbf{\assignment}\\[0.01in]
\textbf{Instructor: Dr.\ Naresh Manwani}\hfill\textbf{\duedate}\\
\smallskip\hrule\bigskip

\begin{thmtype}
\statement
\end{thmtype}

\section{Spanning Set}
  \begin{lemma}
  Let $S_1$ be a spanning set of $\mathbf{W}$ \subset $\mathbf{V}$. Suppose $S_2$ spans $S_1$ , then $S_2$ spans $\mathbf{W}$.
  \end{lemma}
  \begin{proof}
  $$ \mathbf{W} \subset \text{span($S_1$)}$$
  $$\text{$S_1$} \subset \text{span($S_2$)}$$
  $\text{By definition of span,}$
  $$\text{span(span($S_2$))=span($S_2$) }$$
  $\therefore$ $S_2$ spans $\mathbf{W}$
  \end{proof}
  
  \begin{lemma}
  Let S=\{$\vec{a_1},\dots,\vec{a_n}\} \text{ be a spanning set of \mathbf{W}. } Then\ S^{'}\ =\ S\backslash\{\vec{a_i}\} \cup \ \{c\vec{a_i}+\Sigma^n_{j=1 j \neq i}\vec{a_j}\},\\c \neq 0,c \in \mathbf{    F}\ spans\ \mathbf{W}.$
 \end{lemma}
  \begin{proof}
  To prove it,it's sufficient to show that $S^'$ spans S by using previous lemma.
 $\\\\Clearly,$
 $$\text{ $\vec{a_j}$_{\ j \neq i}, \in\ $S^'$ $\subset$ span{$S^'$}}$$
  $Let,$
  $$\text{ $\vec{a^{'}} = \{c\vec{a_i}+\Sigma^n_{j=1 j \neq i}\vec{a_j}\} \in S^{'} $}$$
  $Also,$
  $$\text{ $\vec{a_i} = c^{-1}(\vec{a^{'}}-\Sigma^n_{j=1 j \neq i}c_{j}\vec{a_{j}})\ as\ c \neq 0,c_{j} \in \mathbf{F}$}$$
 $ \text{As $\vec{a_i}$ can be written as linear combination of vectors $\in$ $S^'$}$
 $\\$
 $\text{By definition of span,}$
  $$\text{$\therefore$ $\vec{a_i}$ $\in$ span($S^{'}$)}$$
  $$\text{$\Rightarrow$ $S^'$ spans S}$$
  $\therefore$By using previous lemma
  $S^'$ spans $\mathbf{W}$
  \end{proof}
  
 \begin{lemma}


Let \textbf{S} be a set of \( n \) vectors which span \textbf{W} $\subseteq$ \textbf{V},then the size of any linearly independent subset of \textbf{W} is
$\leq$ \( n\)
\end{lemma}
\begin{proof}
Let
   \textbf{R} = \{ $\bar{\beta_1},\bar{\beta_2},.......,\bar{\beta_n}$\} $\subseteq$ \textbf{W}
   be a set of \( m\) linearly independent vectors in \textbf{W} ( \( m\) $>$ \( n\)) 
   \begin{center}
       \textbf{R} = ( \textbf{R} $\cap$ \textbf{S}) $\cup$ 
       (\textbf{R} $\setminus$ ( \textbf{R} $\cap$ \textbf{S}))
\end{center}
       Suppose ,   $|( \textbf{R} \cap \textbf{S} )|$=\( j\)
   \\    
   \begin{center}
       
   \textbf{R} $\cap$ \textbf{S} = \{ $\bar{\beta_1},\bar{\beta_2},.......\bar{\beta_j}$\}
    \end{center}
 $Let,$
 \begin{center}
  $\bar{\beta_{j+1}}$ $\varepsilon$  (\textbf{R} $\setminus$ ( \textbf{R}
 $\cap$ \textbf{S}))
 \\
 \end{center}
 $Also,$
 \begin{center}
     
 
  (\textbf{R} $\setminus$ ( \textbf{R}          $\cap$ \textbf{S})) $\subseteq$ \textbf{S} 
 \\
 $\Longrightarrow$  ( \textbf{R}          $\cap$ \textbf{S})  $\subset$ \textbf{S}  or  ( \textbf{R}          $\cap$ \textbf{S})  = \textbf{S}
 
 
\end{center}
 \textbf{case \( 1\): ( \textbf{R}          $\cap$ \textbf{S})  = \textbf{S}}
 
 \begin{center}
     
  $\Longrightarrow$ $\because$ $\bar{\beta_{j+1}}$ $\varepsilon$ \textit{span}( \textbf{S})   ,   $\bar{\beta_{j+1}}$ $\varepsilon$ \textit{span}( \textbf{R}
 $\cap$ \textbf{S})
 \newline
 \\$\Longrightarrow$ $\bar{\beta_{j+1}}$ can be expressed as linear combination of set  ( \textbf{R}
 $\cap$ \textbf{S})
 \\
 $\therefore$ any set containing set ( \textbf{R}
 $\cap$ \textbf{S}) and $\bar\beta_{j+1}$ can not be a linearly independent set
 \newline
 \\$\Longrightarrow$ \textbf{R} is not a linearly independent set
\end{center}
\textbf{case \( 2\) : ( \textbf{R}          $\cap$ \textbf{S})  $\subset$ \textbf{S}}
\begin{center}
    

let
   \textbf{S} = \{ $\bar{\beta_1},\bar{\beta_2},.......\bar{\beta_j},\bar{\alpha_{j+1}},....\bar{\alpha_n}$\} $\subseteq$ \textbf{W} , where \{ $\bar{\alpha_{j+1}},\bar{\alpha_{j+2}},.......\bar{\alpha_n}$\} = 
(\textbf{S} $\setminus$ ( \textbf{R}
 $\cap$ \textbf{S}))
 \newline
 \\
 $\because$ \textbf{W} $\subseteq$ \textbf{S} and $\bar{\beta_{j+1}}$  $\varepsilon$ 
 \textbf{W} , $\bar{\beta_{j+1}}$ $\varepsilon$ \textit{span}(\textbf{S})
\newline
 \\

 $\bar{\beta_{j+1}}$ = 
$\sum_{i=1}^{j} {c}_i\bar{\beta_i}$
 $\sum_{k=j+1}^{n} {c}_k\bar{\alpha_k}$
\end{center}


Since R is linearly independent set, $\exists$ atleast one $c_k$ $\neq$ 0,  \ \ \ (k $\in$ [ j+i, n])
\\ Let j $\in$ [ j+1, n] such that $c_{j+1}$ $\neq$ 0,
$$ \text{$\implies$ $\vec{\beta_{j+1}}$ = $c_{ji}$ $\vec{\alpha_{ji}}$ + (  $\sum_{i=1}^{j}$ $c_{i}$ $\vec{\beta_{i}}$ \ + $\sum_{k=j+1\ k\neq ji}^{n}$$c_{k}$ $\vec{\alpha_{k}}$\ \ ($c_{ji}$ $\neq$\ 0 \ )}$$
By lemma 2,
\\
 $$\text{$S_{1}$ = S\ $\backslash$\ \{$\vec{\alpha_{ji}}$\}$\ \cup$\ \{$\vec{\beta_{j+1}}$\} \ spans\ $\mathbf{W}$.}$$
Use $S_{1}$ in place of S and keep on repeating this process,
$$\text{R=(S\ \cap \ R )\ \cup  \ (R\ $\backslash$ S $\cap$ R)\  \ (where \ S \ is \ now \ $S_{1}$)}$$
By induction at some point we will get R $\cap$ S = S
\\ From case 1, R is not a linearly independent set.
\\
\\
Hence Proved.
  \end{proof}
  
  
  

  
  
  
  
  
  

\section{Finite Dimensional Vector Space}
A vector space $\mathbf{V}$ is said to be a finite dimensional vector space if it contains a finite basis.
% \section{Subspace}

\begin{theorem}
Any two basis of a finite dimensional vector space contains the same number of elements.
\end{theorem}

\begin{proof}
Let $\mathbf{B_1} = \{ \vec{0} , \vec{r_2} , \vec{r_3} , .. , \vec{r_n}\}\text{ be a basis(finite)\newline}. $


\begin{itemize}
    \item \text{Note that, } $\mathbf{B_1}$ \text{ spans } $\mathbf{V} \Rightarrow \text{Any other set of linearly independent vectors in } $\textbf{V}$ \text{ can't have cardinality}>\text{n (from Lemma 3.)} $ 
    \item \text{Also note that, any other basis } $\mathbf{B_2} \text{ has only m}<=n\text{ elements. }$$\mathbf{B_2} \text{ is also a spanning set.}$
    $$|\mathbf{B_1}|=n<=m \text{ and } |\mathbf{B_1}|=n<=m \Rightarrow n=m$$
    
\end{itemize}
Thus \text{number of elements in } $\mathbf{B_1}$ and $\mathbf{B_2}$ is same.
\end{proof}
\pagebreak

\begin{thebibliography}{1}
\bibitem{1}
Linear Algebra  by  Kenneth Hoffman and Ray Kunze, 2nd edition, Prentice Hall of India Private Limited, 2006.

\end{thebibliography}

\end{document}